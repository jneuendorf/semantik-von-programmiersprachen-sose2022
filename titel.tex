\title{
    Semantik von Programmiersprachen \\[6pt]
    \large Vorlesung SoSe 2022 \\
    Wolfgang Mulzer
}
\author{Jim Neuendorf}
\maketitle
\begin{abstract}
    \noindent Diese Vorlesung vermittelt Techniken zur Formalisierung der Semantik (Be"-deu"-tungs"-inhalte) von Programmiersprachen. Zunächst werden unter"-schied"-liche Forma"-li"-sie"-rungs"-ansätze (die operationelle, denotationelle und axio"-ma"-ti"-sche Semantik) vorgestellt und diskutiert. Anschließend wird die mathe"-ma"-ti"-sche Theorie der semantischen Bereiche behandelt, die bei der deno"-tatio"-nel"-len Methode, Anwendung findet. Danach wird schrittweise eine umfassende, impe"-rative Programmiersprache entwickelt und die Semantik der einzelnen Sprach"-ele"-mente denotationell spezifiziert. Dabei wird die Fortsetzungstechnik (con"-tinua"-tion sem) systematisch erklärt und verwendet. Schließlich wird auf die Anwendung dieser Techniken eingegangen, insbesondere im Rahmen des Com"-piler"-baus und als Grundlage zur Entwicklung funktionaler Programmier"-spra"-chen.
\end{abstract}